\documentclass{beamer}
\usepackage[utf8]{inputenc}
\usepackage{graphicx}
\usepackage{listings}

\title{Implementierung des Lightway IP Stacks auf Embedded Systemen}
\author{Martin}
\date{\today}

\begin{document}

\frame{\titlepage}

% Slide 1
\begin{frame}
\frametitle{Einführung}
\begin{itemize}
    \item Einführung in die Implementierung des Lightway IP Stacks auf dem ITS-Board
    \item Ziel: Demonstration der Integration und Konfiguration eines TCP/IP-Stacks auf einem ressourcenbeschränkten Mikrocontroller
    \item Relevanz: Essentiell für Netzwerkanwendungen in Embedded-Systemen
\end{itemize}
\end{frame}

% Slide 2
\begin{frame}
\frametitle{Lightway IP Stack: Übersicht}
\begin{itemize}
    \item Leichtgewichtige TCP/IP-Implementierung, ideal für ressourcenbeschränkte Systeme
    \item Unterstützt grundlegende Netzwerkprotokolle: TCP, UDP, ICMP, DHCP
    \item BSD-Lizenz ermöglicht freie Nutzung und Anpassung
\end{itemize}
\end{frame}

% Slide 3
\begin{frame}
\frametitle{Modularer Aufbau des Stacks}
\begin{itemize}
    \item Modulare Architektur für flexible Anpassungen an verschiedene Hardware-Ressourcen
    \item Parameter wie Speicherverbrauch und Verarbeitungsgeschwindigkeit konfigurierbar
    \item Praktisch für verschiedene Embedded-Umgebungen ohne Betriebssystem
\end{itemize}
\end{frame}



% Slide 5
\begin{frame}[fragile]
\frametitle{Konfigurationsdateien im Detail}
\begin{itemize}
    \item \texttt{cc}: Compiler- und Plattform-spezifische Anpassungen
    \item \texttt{sys\_arch}: Systemabhängige Funktionen wie \texttt{sys\_now()} für Zeitmessungen
    \item \texttt{lwip\_interface}: Stellt das lwIP-Interface zur Applikation bereit
    \item \texttt{ethernetif}: Bietet Hardware- und Interrupt-Ansteuerung
\end{itemize}
\end{frame}

% Slide 6
\begin{frame}
\frametitle{Bare-Metal Implementierung}
\begin{itemize}
    \item Betrieb ohne Betriebssystem (NoSys), direkte Hardware-Interaktion
    \item Konfiguration für Bare-Metal: Kein Thread- oder Task-Management notwendig
    \item Notwendig: Speichergröße definieren und Speicher effizient allokieren
\end{itemize}
\end{frame}

% Slide 7
\begin{frame}[fragile]
\frametitle{Netzwerkkonfiguration und Initialisierung}
\begin{itemize}
    \item MAC- und IP-Adresse festlegen
    \item Beispiel für IP-Konfiguration:
\end{itemize}
\begin{lstlisting}[language=C]
#define LWIP_IPADDR "192.168.0.1"
#define LWIP_NETMASK "255.255.255.0"
#define LWIP_GATEWAY "192.168.0.254"
\end{lstlisting}
\end{frame}

% Slide 8
\begin{frame}
\frametitle{Initialisierung des Netzwerkinterfaces}
\begin{itemize}
    \item \texttt{ethernetif\_init()} Funktion für Hardware-Initialisierung
    \item Zuweisung der MAC-Adresse und Start der Hardware-Komponenten
    \item Konfiguration von Interrupt-Prioritäten für eine effiziente Kommunikation
\end{itemize}
\end{frame}

% Slide 9
\begin{frame}[fragile]
\frametitle{Verbindung mit dem Netzwerkinterface}
\begin{itemize}
    \item Interface hinzufügen: \texttt{netif\_add()} und \texttt{netif\_set\_default()}
    \item Konfiguration des Netzwerkinterfaces:
\end{itemize}
\begin{lstlisting}[language=C]
netif_add(&lwip_netif, &ipaddr, &netmask, &gateway, NULL, ethernetif_init, netif_input);
netif_set_default(&lwip_netif);
\end{lstlisting}
\end{frame}

% Slide 10
\begin{frame}
\frametitle{Interrupt Handling für Ethernet}
\begin{itemize}
    \item \texttt{ethernetif\_input()} und \texttt{ethernetif\_init()} Routinen zur Verarbeitung von Paketen
    \item Low-Level-Initialisierung kopiert Datenpakete in eine P-Buff-Struktur zur Weiterverarbeitung
    \item Wichtig: Optimierte Nutzung von Ressourcen durch effizientes Interrupt-Management
\end{itemize}
\end{frame}

% Slide 11
\begin{frame}
\frametitle{Ping-Test und Netzwerkverbindung überprüfen}
\begin{itemize}
    \item Testmethoden wie Ping, um die Stabilität und Reaktionszeit des Netzwerks zu validieren
    \item Regelmäßige Überprüfung des Netzwerkstatus mit \texttt{sys\_check\_timeouts()} und anderen Timer-Funktionen
\end{itemize}
\end{frame}

% Slide 12
\begin{frame}
\frametitle{Zusammenfassung und Ausblick}
\begin{itemize}
    \item Erfolgreiche Implementierung des Lightway IP Stacks auf dem ITS-Board
    \item Effiziente Bare-Metal-Konfiguration ermöglicht ressourcenschonende Netzwerkkommunikation
    \item Weitere Optimierungen und detaillierte Anpassungen an die Hardware für den praktischen Einsatz
\end{itemize}
\end{frame}

\end{document}